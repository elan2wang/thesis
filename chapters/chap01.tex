\chapter{绪论}

\section{研究背景及意义}

\begin{itemize}
\item 传统的新闻传播方式的弊端,数据量之大,有效信息获取非常困难
\item 新技术的出现,是研究内容变得可行
\item 如果采用新的技术,人们可以快速有效,直观的了解信息
\end{itemize}

主题模型在机器学习和自然语言处理等领域是用来在一系列文档中发现抽象主题的一种统计模型。近年来,主题模型已经被广大的学者和研究人员应用在了文本信息挖掘、文本分类、图像分类和理解、情感分析、学术文章挖掘、语义分析、推荐系统等诸多方面 \cite{leskovec2009meme, wang2012tm, 008:labeledlda, 2011:xuge, 011:mblda}。
情感分析(opinion mining,  sentiment analysis)是近年来新出现地一个研究方向,其基本任务就是从用户生成的包含观点和意见的文本中抽取出这些观点和意见,然后生成情感摘要,进行情感分类,自动构建情感词典等等情感分析任务。主题模型在情感分析最主要的任务就是学习出来用户讨论和用户评论中的内容主题。在情感分析中每一个topic通常被称为aspect,在情感分析中,会将词汇区别为情感词汇和主题词汇。
学术文章挖掘是主题模型的一个重要应用,通过对学术文章的挖掘来进一步理解研究领域的发展和进化,对于了解之前的科研成果和未来的发展趋势有非常重要的意义。对于学术文章语料集合,LDA模型并没有考虑到文章的作者,实际上是把所有的作者都看作完全等同的。Rosen-Zvi等人(2004) \cite{rosen2004author} 提出了作者主题模型 (Author-Topic Model, ATM),该模型假定每一个作者都对应一个主题分布,文档生成过程中,先随机选取一个作者,然后根据该作者的主题分布确定下一个词的主题。ATM模型假定,不同作者当选定同一主题后,他们写文章时使用的词都是一样的,显然这不符合实际情况。针对这个问题,Kim等人(2012) \cite{Kim:2012} 提出了实体主题模型ETM(Entity Topic Model),该模型引入层次的Dirichlet先验,将词与主题分布的先验、词与实体分布的先验通过线性组合,作为与主题实体对分布的先验,这样每个词的生成都与主题实体对相关。


现在随着Twitter, Weibo等社交媒体的快速发展,对社交媒体的主题挖掘也成了一个热点的研究问题,目前主要应用的模型有LDA \cite{Blei:2003}, Author-Topic Model\cite{rosen2004author} 和Twitter-LDA\cite{zhao2011comparing}, MB-LDA\cite{011:mblda}等等。

\section{国内外研究现状}

\begin{itemize}
\item 信息检索技术的发展
\item 主题模型等文本挖掘技术,以及扩展模型
\item 新闻主题挖掘相关技术比较
\end{itemize}

主题模型(Topic Models) \cite{steyvers2007probabilistic} 的基本思想是,一个文档是由多个主题混合而成的,而主题是在词库上的一个概率分布。主题模型是一个生成式模型,为了生成一个文档,首先选择一个主题的概率分布,然后对于文档的每个词,根据主题的概率分布随机的选择一个主题,并从该主题中选择一个词。利用统计学的一个方法,我们可以推理出生成该文档集合的主题集合。


主题模型的起源是隐性语义索引(Latent Semantic Indexing, LSI) \cite{deerwester1990indexing}。LSI并不是概率模型,因此也算不上一个主题模型,但是其基本思想为主题模型的发展奠定了基础。在LSI基础上,Hofmann(1999; 2001) \cite{hofmann1999probabilistic, hofmann2001unsupervised} 提出了pLSI(Probabilistic Latent Semantic Indexing)模型,但pLSI并没有用一个概率模型来模拟文档的产生,只是通过对训练集种的有限文档进行拟合,得到特定文档的主题混合比例。这样就导致了pLSI模型参数随着训练集中的文档数目线性增加,出现过拟合现象,而且对于训练集以外的文档很难分配合适的概率。Blei等人(2003) \cite{Blei:2003} 在pLSI基础上加以扩展,提出了LDA(Latent Dirichlet Allocation)模型。LDA模型用服从Drichlet分布的K维隐含随机变量表示文档的主题混合比例,来模拟文档的产生。Dirchlet分布作为多项分布的共轭先验,很好的简化了统计推理问题。LDA模型中,主题数T通常是根据我们的先验知识随机选取的,Griffiths和Steyvers(2004) \cite{griffiths2004finding} 通过引入标准Bayes方法,将主题数的确定转变成一个模型选择问题,从而确定最佳的主题数,并且作者提出了使用MCMC和Gibbs采样的方法来进行参数评估。


LDA模型的其中一个局限是,它难以描述主题间的关联关系。而在许多文本语料中,隐主题之间本就有很强的关联性,比如:在一个关于Science的语料中,有关基因的文章也有可能是有关健康和疾病的,而不太可能是关于天文学的。产生该局限性的原因是,在主题的Dirichlet分布中,我们假设各个分量几乎是独立的。Blei和Lafferty(2007) \cite{lafferty2005correlated} 提出了CTM(Correlated Topic Model)模型,该模型通过Logistic Normal Distribution来描述主题的分布。相比LDA,CTM模型能够更好的表达真实语料中主题的分布,然而模型复杂化之后,计算复杂度也随着增大。
不管是LDA还是CTM都假设语料库中的文档是可交换的,但是在许多实际的语料中,该假设并不合适,如,学术期刊,邮件,新闻等等的内容,都是随着时间不断演化的。为了显示地描述和发现主题的动态变化情况,Blei和Lafferty又提出了DTM(Dynamic Topic Model) \cite{Blei:2006} 。在DTM模型中,作者按时间片对文档集合进行划分,然后分别对每个时间片内地文档用LDA模型进行建模,而时间片t的主题是从时间片t-1的主题进化而来的。和CTM模型一样,作者采用Logistic Normal Distribution来描述时序主题的不确定性。但是对于一些应用,可能对于事件的粒度要求更高或者很难划分数据,X. Wang(2006) \cite{wang2006topics} 提出了一个非马尔科夫连续时间模型,该模型除了文本信息以外,将时间标签也作为可见信息,
然后通过主题分布信息同时关联起来词汇和事件标签。但是该模型假定主题集合不随时间变化而变化,C. Wang(2008) \cite{wang2008continuous} 进一步松弛了这种假设,提出另一种连续时间主题模型,在这个模型中,主题集合随着时间变化而变化。


LDA从提出到现在已经被广泛应用于文本挖掘和信息处理领域。并且现在也出现了许多关注主题模型性能的工作,这说明主题模型已经不局限于理论研究阶段,它的实用性得到了认可,因此需要更高效的训练算法。Mariote等人(2007)\cite{mariote2007parallelized} 提出了变分EM(Variational EM)算法来对训练过程进行加速,以便应用于多处理器和分布式环境。Asuncion等人(2008)给LDA模型和HDP模型提出了分布式算法在保证全局正确性的前提下,各个处理单元能够独立进行Gibbs采样。Hoffman等人(2010) \cite{hoffman2010online} 提出了LDA模型的在线变分贝叶斯方法(Online Variational Bayesian)。

\section{本文的主要研究内容}
本文着眼于针对某一新闻事件的所有报道组成的语料库,进行信息的深度挖掘,构建主题随时间变化的演化图,同时通过可视化方法将事件中人物之间的交互展现出来。因此本文的研究内容包括两个方面,其一是采用主题模型挖掘时序文档的主题演化,其二是研究文本可视化相关方法。新闻报道的六要素为:事件的起因,经过,结果,时间,地点,人物。研究内容一是为了挖掘事件的发展变化(即事件的起因,经过和结果),而研究内容二则是为了刻画事件中人物之间的关系,两者结合在一起,我们便能对该新闻事件有较为清晰的认识,我们将该任务称之为\textbf{新闻故事线抽取}。


新闻报道有很强的实效性,并且在不同的时间段,对于同一事件的报道往往有较大的差异,然而在同一时间段内,大部分的报道则有很强的相似性,本文结合新闻报道的这种时序特性,提出了一种基于主题模型的新闻主题变化抽取方法,该方法能够很好的捕捉事件发展的主要阶段,并挖掘各个阶段内事件发展的最新动态。


新闻故事的另一个主要的特征是人物,为了刻画故事中人物之间随时间变化的交互关系,本文基于 "Movie Narrative Charts" \cite{xkcd657} 提出了 Storyline 用于描述新闻事件中人物之间的交互关系,并提出了一套自动从文本中抽取 Storyline 的方法框架,并就该方法框架中的各个步骤提出了相应的算法实现。

\section{本文的组织结构}
本文共分为五个章节,各章节的主要内容如下:


第一章绪论:介绍文本的研究背景和意义,国内外研究现状,本文的主要研究内容,最后给出本文的组织结构。


第二张相关工作:介绍本文中用的一些理论和方法,详细阐述了主题模型(如:LDA,DTM)的原理和参数推理方法,以及它们的应用场景。另一方面,介绍了在文本可视化中常用的一些方法和技术,如:标签云,单词树,星系视图,主题山地,主题河流等等。


第三章新闻事件的主题演化分析。本章重点阐述了新闻报道的时序分布特点,并提出了自动划分新闻语料的方法,进一步结合时序信息的改进主题模型,并将该方法用于挖掘新闻事件的主题演化。最后通过实验验证了本文方法。


第四章新闻事件的可视化方法设计。本章终点阐述 Storyline 设计原则和元素定义,并通过形式化语言描述了 Storyline 构建方法,针对构建的各个步骤提出了有效的算法进行求解。


第五章总结与展望:本章总结本文所做的工作,提出接下来亟待解决的问题,并展下一步的研究方向。

\section{本章小结}
本章介绍了本研究的背景和意义,在当今信息爆炸的现状下,如何快速有效的获取信息成为了一个棘手的问题。主题模型作为一种文本信息挖掘的有效方法,在过去的十几年中获得广泛的关注和较快的发展。各国研究者从不同的研究方向和研究方法上着手,提出了各种各样的主题模型,它们或针对特定文本进行深度挖掘,或采用的不同的方法进行模型的参数学习,或采用分布式或在线学习的方法提高模型的运算速度。基于已有的研究,本文提出了针对新闻语料的主题演化分析方法研究,并介绍了本文的主要研究内容和本文的组织结构。