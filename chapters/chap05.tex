\chapter{总结与展望}

\section{本文的工作}
新闻作为一种获取信息的来源,在人们的生活中起着重要的作用。尤其是针对一些重大新闻的相关报道,
不仅是作为一种信息的传递,更重要的是它们会引导社会舆论的走向,并对社会生产生活的方方面面产
生不同程度的影响。因此,如何快度有效的跟踪事态的发展和演化,以及事件中人物之间的交互关系显
得十分重要。为了解决这一问题,本文从新闻事件的两个主要方面作为切入点深入挖掘新闻事件,并向
读者呈现一个完整故事线。

首先,本文通过分析特定事件的相关报道的时间分布,发现了新闻报道的时序分布特点,即:在事件的
整个发展历程中,与之相关的报道数量在时间轴上呈现出多个波峰和波谷,波峰和波谷交替出现。基于
这一发现,我们尝试将这一信息与主题模型进行结合,提出了一套能够跟踪主题演化的模型,用于挖掘
新闻事件的发展变化。该模型相对其它动态主题模型,更适合针对某一特定主题的语料库,能够挖掘该
主题下更深入的一些子主题变化情况。在第三章的一开始我们提出了两点假设,分别是:对于同一个新
闻事件的报道往往呈现出多个阶段,且每个阶段内报道的文章都是针对的同一个方面,文章件有很高的
相似性;如果语料库中文本之间的一致性越高,那么用主题模型分析,挖掘出的信息也就越丰富。我们
通过实验验证了这两个假设,并更一步说明了我们模型的有效性。

其次,本文通过可视化的方式去呈现事件发展中人物之间的交互关系。本文采用storyline方法来呈
现人物之间的交互关系,并提出了一套高效的混合优化算法来提高storyline的生成速度和优化布局。
已有的生成算法采用启发式优化算法,如:遗传算法等,来进行布局的优化。然而这些方法存在两个非
常普遍的通病:由于启发式算法是一个迭代算法,并且启发式策略比较盲目,因此算法的时间复杂度非
常高,往往要几分钟甚至几十分钟才能生成最终布局;另一个是容易陷入局部最优。本文的方法将优化
目标进行拆解,并且通过多个步骤逐步进行优化,有效的降低了算法的时间复杂度;并且通过初始布局
的选择,使得算法尽可能的能够找到更优的布局。

通过以上的两个方面的研究,我们能够很好地为读者呈现一个事件发展过程完整故事线,包括了事件的
发展动态(主题演化)和人物之间随时间的交互关系。
\section{下一步工作展望}
虽然通过上面完成的几个任务,很好的解决了新闻事件故事线自动生成。但是每项任务仍然还有很大的
空间可以进行探索和挖掘。

对于新闻主题演化跟踪中,我们紧紧研究了主题本身之间的演化关系,并没有很好的将事件中人物对主
题演化的影响融入模型当中。在接下来的工作中,我们需要探索如何将人物这一因素结合到主题模型当
中,在文档的生成过程中加入人物的影响。

对于storyline的生成算法,我们将优化问题拆解成了几个部分,每个部分优化其中一个目标,这样
虽然有效的降低了模型的时间复杂度,提高了算法运行效率;但是,分开优化各个指标时忽略了它们之
间的相互影响.比如,在对齐过程中可能会影响排序过程中获得结果,使得线条之间产生不必要的聚合。
因此,我们需要在后续的工作中,提出了一个有效的平衡因子来权衡它们之间的影响。另外,我们还打
算基于该算法框架开发一套更加完善的storyline在线生成系统,提供用户更好的用户体验和更好的
交互方式,方便用户自己生成任何新闻事件的storyline。