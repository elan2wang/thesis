\chapter{相关工作介绍}
\section{主题模型介绍}

\subsection{LDA}
\label{intro-lda}

\subsection{DTM}
\label{intro-dtm}

\section{主题模型的应用}

\subsection{主题模型在文本分析中的应用}
\subsection{主题模型在社交网络中的应用}

\section{文本可视化相关方法}
\subsection{单文本内容可视化}
\subsubsection{标签云}
标签云(tag cloud)又称为文本云(text cloud)

\subsubsection{单词树}
单词树(world tree)

\subsection{多文挡可视化}
\subsubsection{星系试图}
星系试图(galaxy view)

\subsubsection{主题山地}
主题山地(themescapes)

\subsubsection{新闻地图}
新闻地图 (newsmap)

\subsection{时序文本可视化}
对于时序文本,除了要呈现语料库的静态主题信息以外,更重要的是如何更好得对动态的主题变化进行可视化展现,而这也正是吸引广大研究人员的地方。下面将列举几个比较有代表性的集中方法:
\subsubsection{Theme River}
Theme River \cite{Havre:2000}是用于模拟主题随时间变化的可视化方法。主题河流通过不同颜色的色带表示不同的主题,色带的宽度表示主题的强度,色带越宽表示该主题在该时间段内的强度越强;反之,色带越窄表示主题在该时间段的影响越小。

\subsubsection{Metro Map}
Metro Map \cite{shahaf2012trains} 的想法缘自地铁的网络图,它是由一系列代表事件线索的线条组成,每个线条代表了事件的一个方面,线条的交叉和重叠部分表示不同线索之间的关联和融合。通过 Metro Map 用户可以很快的获得一个故事的整体情况。
