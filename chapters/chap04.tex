\chapter{新闻事件的可视化方法研究}

\section{文本可视化相关方法}
\subsection{单文本内容可视化}
\subsubsection{标签云}
标签云(tag cloud)又称为文本云(text cloud)

\subsubsection{单词树}
单词树(world tree)

\subsection{多文挡可视化}
\subsubsection{星系试图}
星系试图(galaxy view)

\subsubsection{主题山地}
主题山地(themescapes)

\subsubsection{新闻地图}
新闻地图 (newsmap)

\subsection{时序文本可视化}
对于时序文本,除了要呈现语料库的静态主题信息以外,更重要的是如何更好得对动态的主题变化进行可视化展现,而这也正是吸引广大研究人员的地方。下面将列举几个比较有代表性的集中方法:
\subsubsection{Theme River}
Theme River \cite{Havre:2000}是用于模拟主题随时间变化的可视化方法。主题河流通过不同颜色的色带表示不同的主题,色带的宽度表示主题的强度,色带越宽表示该主题在该时间段内的强度越强;反之,色带越窄表示主题在该时间段的影响越小。

\subsubsection{Metro Map}
Metro Map \cite{shahaf2012trains} 的想法缘自地铁的网络图,它是由一系列代表事件线索的线条组成,每个线条代表了事件的一个方面,线条的交叉和重叠部分表示不同线索之间的关联和融合。通过 Metro Map 用户可以很快的获得一个故事的整体情况。

\section{Storyline 方法研究}
Storyline 是指描述的是人物之间随时间变化的社交关系,它将这些复杂的信息在一张图形上展现出来。该可视化方法最早是由 Mounroe 提出,并发布在著名漫画网站 XKCD 的 "Movie Narrative Charts" \cite{xkcd657} 中。在这个连载漫画中,手绘的 Storyline 用于概括电影中角色之间的互动。Figure.\ref{xkcd} 正是摘自 XKCD 网站。
% BEGIN == XKCD 手绘图
\begin{figure}[htb]
	\centering
		\includegraphics[width=15cm]{xkcd_jurassic_park}
	\caption{Jurassic park 人物关系叙事图,X轴表示时间,每个线条分别代表一个角色}
	\label{xkcd}
\end{figure}
% END == XKCD 手绘图


\subsection{相关研究}
@Todo 介绍 storyline 出现之前的一些展现故事的可视化方法,以及由 storyline 延伸来的可视化方法。

\subsection{设计原则}
在介绍 storyline 的设计准则之前,我们首先定义 storyline 中最基本的一对映射关系:
\begin{itemize}
\item \textbf{角色(Character 或 Entity)}
\item \textbf{线条(Thread)}
\end{itemize}
\textbf{Character} 是指故事中出现的人物或其他个体。新闻语料中,角色可以是人物、组织机构、国家等命名实体,它们一起构成了新闻故事六要素(时间、地点、人物、起因、经过和结果)之一的人物。电影、小说和新闻事件等通常都会有一个\textbf{主角(Leading Character)} 和若干的\textbf{配角(Ordinary Character)},所有的故事都围绕着主角展开和发展,并通过配角来充实整个故事。为了将故事中的 \textbf{Characters} 以及它们之间发生的事件映射到 storyline 中,便得到了 storyline 设计的基本原则:
\begin{itemize}
\item 水平轴上从左往右的\textbf{线条}表示故事中一个\textbf{角色}完整的生命周期。
\item 多个 \textbf{线条} 在某一时间段聚拢到一块,表示有这些\textbf{角色}共同参与了一个事件。线条之间的\textbf{收敛}和\textbf{发散}分别表示角色间的事件的开始和结束。
\end{itemize}
\textbf{Thread} 作为 \textbf{Characters} 的载体,每一个 \textbf{Thread}  都可以看作是故事的维度,一个由 \textbf{Threads} 组成的集合以及它们之间的位置随时间的变化(如:\textbf{收敛 Convergence},\textbf{发散 Divergence} 和\textbf{交叉 Intersection})描述了整个故事的发展变化。以上设计原则可以指导我们设计出更美观且易于理解的图形,因此被许多研究者广泛采用\cite{Ogawa:2010, Kim:2010},并在构建特定领域的 storyline 上获得了一定程度上的成功。然而,这些以上原则并没有很好的将文本的上下文信息(如,位置)表现出来。为了更好的描述事件发生的位置信息,Tanahashi 和 Ma \cite{tanahashi2012design} 在storyline 中用一个封闭的轮廓线作为背景将事件圈起来,并用不同的颜色表示不同的事件。同时为了让 storyline 展现得更加美观并增强可读性,Tanahashi 和 Ma 还添加了一条新的标准:
\begin{itemize}
\item 线条只有在收敛和发散的时候才可以偏离原始的方向。
\end{itemize}
也就是说,要尽可能让线条保持直线,减少线条的摆动。本文中,我们将采用以上这些原则来设计我们的 storyline。

\subsection{最优化度量}
\label{metrics}
给定一个角色(实体,人物)的集合以及它们之间在时间上的关联关系,我们的目标是要构建出一个美观且易于理解的 storyline 可视化图形。图形布局的构建可以分解为一系列的最优化求解问题,在我们的布局最优化方法中,我们采用 Tanahashi 和 Ma \cite{tanahashi2012design} 提出的最优化度量标准来定义我们的优化目标。
\begin{itemize}
\item \textbf{线条交叉 Line Crossovers}:最小化线条交叉的数量。
\item \textbf{线条摆动 Line Wiggles}:增强线条的笔直和连续性。
\item \textbf{空白距离 White Space Gaps}:最小化空白空间之间的距离,增强紧凑度。
\end{itemize}
很显然\textbf{线条交叉}容易造成图形的混乱,所以应该最小化交叉的数量。\textbf{线条摆动}是指线条偏离了原始的方向,这会打断线条的连续性,因此减少线条摆动次数可以增强可读性。\textbf{空白距离}是指图形中两个可视元素之间的空白空间的大小,空白空间是图形中区分元素的必要组成部分,然后过度的距离不仅容易造成空间的浪费,更重要的是容易让读者误以为这是上下文中角色之间的语意距离,所以也应当减少不必要的空白空间。

\section{Storyline 数据定义}
在 Storyline 的可视化中用到的数据最基本的形式就是各个角色之间按时间先后顺序排列的交互列表。这些角色之间的一系列交互可以被拆分成一系列\textbf{会话 Session},每个\textbf{会话}代表了一个时间跨度以及这段时间内有交互的角色集合。严格意义上,我们可以将\textbf{会话}定义为包含以下元素的基本单元:
\begin{itemize}
\item \textbf{起始时间}
\item \textbf{时间跨度}
\item \textbf{成员列表}
\end{itemize}
\textbf{起始时间}表示\textbf{会话}从何时开始,\textbf{时间跨度}表示该\textbf{会话}的持续时间,\textbf{成员列表}是指参与本次\textbf{会话}的所有角色的集合。

% BEGIN == JSON Schema & Example
\begin{listing}
\begin{minted}[frame=single, framesep=2mm, tabsize=4]{js}
{
    "sessions": [
        {
            "id": 0,
            "start": 0,
            "duration": 5,
            "characters": [
                0,
                2
            ]
        }
        {
            "id": 1,
            "start": 5,
            "duration": 10,
            "characters": [
                0,
                3,
                10
            ]
        }
   ]
}
\end{minted}
\caption{JSON example} 
\label{json-example}
\end{listing}
% END == JSON Schema & Example

\section{Storyline 布局设计}
布局设计中我们的主要目标就是最小化 \ref{metrics} 小节中提出的三个度量指标:\textbf{Line Crossover number},\textbf{wiggle number},\textbf{white space gaps}。 显然我们很难设计一个方法能够同时最优化以上三个指标,因为这需要在一个巨大的搜索空间内进行全面的尝试和探索。当然,如果不能求得全局的最优解,我们也可以采用\textbf{遗传算法}\cite{tanahashi2012design} 等来求得一个接近最优解的布局。然而,遗传算法的不足是获得的局部最优解经常会比较差,并且容易造成\textbf{早熟收敛}[@Todo reference]的问题。

为了弥补遗传算法的不足,并缩小搜索空间,我们将最优化的过程分解成几个独立的子问题。每个子问题分别最优化 \ref{metrics} 小节中的一个度量指标。子问题处理的原则是重要指标优先处理,因为它们对最终的可视化效果影响最为严重,更重要的是,在优化后续指标时不能影响前一指标的效果。已有研究[@Todo reference]指出减少线条交叉是最重要的步骤,减少线条的摆动以及空白距离次之。基于此,我们将问题分解成以下三个步骤:
\begin{itemize}
\item \textbf{步骤1:减少线条交叉的数目。}影响线条交叉数目最直接的因素就是线条的排列顺序以及会话的位置,因此为了减少线条不必要的交叉数目,我们要做的就是调整他们在空间上的排序以及位置。
\item \textbf{步骤2:减少线条摆动的数目。}线条的摆动收到会话的位置的影响,因此要减少线条的摆动就要尽可能得将会话摆放在适当的位置。同时,保证在同一个时间段内独立的,没有参与任何会话的的线条保持直线。
\item \textbf{步骤3:减少空白空间的距离。}为了使生成的图片更加的美观并且让信息更好的传递,我们还需要连续的优化空间中的空白距离。
\end{itemize}
 
 [@Todo 画图说明以上三个步骤]
 
\subsection{线条排序}

\subsection{元素对齐}
\begin{equation}
E_{align} = max \sum_{t=1}^{n_t-1} H\left(t\right)
\end{equation}
其中$H\left(t\right)$表示在$t$和$t+1$之间笔直线段的数目

\begin{equation}
E_{align} = \sum_{t=1}^{n_t-1} max H\left(t\right)
\end{equation}


\subsection{空间压缩} 
\begin{equation}
min \sum_{i=1}^{n_e} \sum_j^{n_t-1} \left(y_{i,j} - y_{i, j+1} \right)^2 + \beta \sum_{i=1}^{n_e} \sum_{j=1}^{n_t} y_{i,j}^2
\end{equation}
\label{eq:compact}
其中
\begin{subequations}
\begin{align}
	y_{i_1,j} < y_{i_2, j} & \qquad \text{If  } S_{i_1,j} \prec S_{i_2,j}; \label{eq:compact-1}\\
	y_{i,j} = y_{i,j+1} & \qquad \text{If  } S_{i,j} \leftrightarrow S_{i+1, j}; \label{eq:compact-2}\\
	y_{i,j} - y_{i+1,j} = d_{in} & \qquad \text{If  } SID(S_{i,j}) = SID(S_{i+1}, j); \label{eq:compact-3}\\
	\left|y_{i,j} - y_{i+1,j}\right| \geq d_{out} & \qquad \text{If  } SID(S_{i,j}) \neq SID(S_{i+1}, j). \label{eq:compact-4}
\end{align}
\end{subequations}
上面的式子中$S_{i,j}$指表示角色$i$在时间$j$内的线条片断, $y_{i,j}$表示$S_{i,j}$的$y$轴的坐标值;$SID(S_{i,j})$指$S_{i,j}$所在的会话ID。公式 \label{eq:compact} 中的第一项是线条摆动的惩罚函数,第二项是空白空间的惩罚函数。$\beta$ 是这两项之间的一个平衡因子,在我们的实验中,我们取 $\beta = 1$。以上的四个约束条件分别表示:
\begin{itemize}
\item (\ref{eq:compact-1}) 
\item (\ref{eq:compact-2})
\item (\ref{eq:compact-3})
\item (\ref{eq:compact-4})
\end{itemize}











